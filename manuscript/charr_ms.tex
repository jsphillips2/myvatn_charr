% ---------------------------------------------------------------------------------------
% ---------------------------------------------------------------------------------------
% Preliminaries
% ---------------------------------------------------------------------------------------
% ---------------------------------------------------------------------------------------

\documentclass[11pt]{article}
\usepackage[letterpaper, margin=1in]{geometry}
\usepackage{newtxtext,newtxmath}
\usepackage[math-style=ISO]{unicode-math}
\usepackage{fullpage}
\usepackage[authoryear,sectionbib]{natbib}
\linespread{1.7}
\usepackage[utf8]{inputenc}
\usepackage{lineno}
\usepackage{titlesec}
\titleformat{\section}[block]{\Large\bfseries\filcenter}{\thesection}{1em}{}
\titleformat{\subsection}[block]{\Large\itshape\filcenter}{\thesubsection}{1em}{}
\titleformat{\subsubsection}[block]{\large\itshape}{\thesubsubsection}{1em}{}
\titleformat{\paragraph}[runin]{\itshape}{\theparagraph}{1em}{}[. ]\renewcommand{\refname}
  {Literature Cited}
\DeclareTextSymbolDefault{\dh}{T1} % for Icelandic ð symbol:
\usepackage{graphicx} % for figures
\usepackage{booktabs} % for tables
\usepackage{amsmath} % for split math environment
\usepackage{enumitem}




% ---------------------------------------------------------------------------------------
% ---------------------------------------------------------------------------------------
% Title page
% ---------------------------------------------------------------------------------------
% ---------------------------------------------------------------------------------------


\title{Opposing trends in survival and recruitment slow the recovery 
        of an historically overexploited fishery}

\author{
Joseph S. Phillips$^{1,2, \dagger}$ \\
Gu{\dh}ni Gu{\dh}ergsson$^{3}$ \\
Anthony R. Ives$^{2}$
}

\date{}

\begin{document}

\raggedright
\setlength\parindent{0.25in}

\maketitle


\noindent{} 1. Department of Aquaculture and Fish Biology, 
H\'{o}lar University, Skagafj\"{o}r{\dh}ur 551 Iceland

\noindent{} 2. Department of Integrative Biology, 
University of Wisconsin, Madison, Wisconsin 53706 USA

\noindent{} 3. Marine and Freshwater Research Institute, Reykjav\'{i}k Iceland

\noindent{} $\dagger$ E-mail: joseph@holar.is



\bigskip

Running head: {Arctic charr demography}

\linenumbers{}

\clearpage





% ---------------------------------------------------------------------------------------
% ---------------------------------------------------------------------------------------
% Abstract
% ---------------------------------------------------------------------------------------
% ---------------------------------------------------------------------------------------


\section*{Abstract} \label{abstract}

Quantifying temporal variation in demographic rates is a central goal of population ecology,
in both basic and applied settings.
In this study, we analysed a multidecadal age-structured time series 
of arctic charr (\emph{Salvelinus alpinus}) catch  in Lake M\'{y}vatn, Iceland, 
to infer the time-varying demographic response of the population 
to reduced harvest in the wake of the fishery's collapse.
Our analysis shows that while survival probability of adults increased following the alleviation
of harvesting pressure, per recruitment consistent declined over the entire study period.
The countervailing demographic trends resulted 
in no directional change in the total population size or population growth rate.
Rather, the population dynamics were dominated by large interannual variability
and a shift towards a relatively older age distribution.
These results are indicative of a slow recovery of the population following its collapse,
despite the rising number of adults due to relaxed harvest.



\bigskip

\textit{Keywords}: {}

\clearpage



% ---------------------------------------------------------------------------------------
% ---------------------------------------------------------------------------------------
% Main text
% ---------------------------------------------------------------------------------------
% ---------------------------------------------------------------------------------------

\section*{Introduction}

Quantifying temporal variation in demographic rates is a central goal of population ecology,
as this underpins efforts to characterize both exogenous and endogenous drivers of population dynamics 
\citep{twombly1994comparative, zeng1998, koons2016life}.
However, this endeavor is challenging, 
as even the characterization of directional trends in demographic rates requires
data spanning many years and 
often relies on intensive mark-recpature (and related) approaches for statistical inference
\citep[e.g.,][]{forcada2008life, hunter2010climate}.
Both of these challenges are amplified in populations subject to large interannual variation
in demographic rates, 
which are often the targets of both basic and applied interest
\citep[e.g.,][]{white2007irruptive}.
Consequently, 
there is a need for addtional studies that 
explict quantify temporal variation in demographic rates 
and the resulting population dynamics, 
particularly for populations that have not been the subjects 
of intensive mark-recpature-style campaigns.

In this study, we analysed a multidecadal age-structured time series 
of arctic charr (\emph{Salvelinus alpinus}) catch  in Lake M\'{y}vatn, Iceland, 
to infer the time-varying demographic response of the population 
to reduced harvest in the wake of the fishery's collapse.
Arctic charr are salmonids with a circumpolar distribution
and are basis of numerous commerical fisheries \citep{klemetsen2003atlantic}.
Moreover,
arctic charr have important effects in many artic 
and boreal freshwater foodwebs owing to their roles as top consumers
\citep{jeppesen2001fish, klemetsen2003atlantic}.
In M\'{y}vatn, the charr sustained a large commerical fishery throughout much of the 
twentieth century that was subject to exploitation rates upwards of 80\%,
resulting in the fishery's collapse by the late 1980s \citep{gudbergsson2004}.
A monitoring program was instituted in 1986 to monitor the population's
response to declining harvesting pressure over the next few decades,
including catch restrictions imposed in the early 2000s.

Using these monitoring data, 
we parameterized an age-structured demographic model with time-varying recruitment and survival 
to charactreize the dynamics of the population.
We modeled temporal variation in the demgoraphic rates as random walks, 
an approach that had previously been applied 
to infer population growth rates from non-structured abundance 
\citep{zeng1998}
and for age-specific mortality rates inferred from fisheries stock assessments 
\citep{nielsen2014estimation}. 
This method takes advantage of the entire time series for estimating the parameters 
while allowing them to vary smoothly through time. 
Furthermore, the model is able to characterize a range of dynamics including those
arising from negative density-dependence and environmental perturbations.
The latter may be particularly important in the case of the M\'{y}vatn's charr,
as the lake is subject to large fluctuations in primary and secondary production
that may cascade up to the charr population 
\citep{einarsson2004myvatn, einarsson2004clad, gardarsson2004population}.
Accordingly, the purpose of our analysis is to 
(a) characterize interannual variability in survival recruitment 
in M\'{y}vatn's charr population 
and (b) determine whether directional trends in these demographic rates have
resulted in the recovery of the population in the wake of its collapse. 

% ---------------------------------------------------------------------------------------
\section*{Methods} 
% ---------------------------------------------------------------------------------------

% ---------------------------------------------------------------------------------------
\subsection*{Study system} 
% ---------------------------------------------------------------------------------------

Mývatn is located in northeastern Iceland (65°40’N 17°00’W) 
and has a tundra-subarctic climate (Björnsson and Jónsson 2004). 
The lake is 37 km2, divided into north (8 km2) and south (29 km2) basins 
connected by a narrow channel (Einarsson et al. 2004). 
Mývatn is shallow (mean depth = 2.3 m) and is fed by cold and 
warms springs rich in phosphorous 
(south basin N and P inputs of 1.4 and 1.5 g m-2 y-1, respectively) (Ólafsson 1979). 
Consequently, the lake has high primary production 
(Ólafsson 1979) that supports large but variable populations of benthic invertebrates 
such as midges and cladocerans 
(Lindegaard and Jónasson 1979, Einarsson and Örnólfsdóttir 2004, Garðarsson et al. 2004). 
The benthic invertebrates are an important food source for Mývatn’s vertebrate populations, 
including Arctic charr, three-spined sticklebacks (Gasterosteus aculeatus), 
brown trout (Salmo trutta), and waterfowl (Einarsson et al. 2004). 

Arctic charr (Salvelinus alpinus) are salmonids with a circumpolar distribution 
and are important consumers in many boreal and arctic lakes (Klemetsen et al. 2003). 
While some charr populations are anadromous (Klemetsen et al. 2003), 
Mývatn’s population to resides strictly within the lake, 
despite a major outflow that connects the lake to the Greenland Sea (Guðbergsson 2004). 
Spawning occurs in the fall among individuals aged 4 years and older. 
There are two charr morphs within Mývatn: “regular” (size at maturation of 35-50cm) 
and “dwarf” (20-25cm). However, the dwarf morph is restricted to a small 
and relatively isolated part of the south basin; therefore, 
the “regular” morph is the focus of this study. 
Gut content data reveal a diverse diet, including midges, snails, clams, zooplankton, 
benthic crustaceans, and sticklebacks. 
The large-bodied cladoceran Eurycercus is thought to be a particularly important prey item 
(Guðbergsson 2004). 
While some charr populations are cannibalistic (Klemetsen et al. 2003), 
especially in the absence of other large-bodied prey (e.g., fish), 
the charr in Mývatn appear to lack cannibalism. 

% ---------------------------------------------------------------------------------------
\subsection*{Data} 
% ---------------------------------------------------------------------------------------

Systematic surveys of Mývatn’s Arctic charr have been conducted 
with gill nets every year from 1986-2017 by a single researcher (G. Guðbergsson) 
using a consistent methodology. 
The surveys largely took place after the period of most dramatic population decline 
(Guðbergsson 2004). Most captured individuals were measured for length, mass, 
age (either based on otoliths or estimated from size), and a binary index of maturity. 
Surveys were conducted at 12 sites around the lake, 
most of which were located in the south basin. 
Surveys were conducted in fall (late August through September) 
of every year (just before spawning) and in June in a subset of years. 
For this analysis we used only the September data. 
Therefore, in our age classifications an individual of age ‘x’ is an individual 
that survived to that age (e.g., an “age 1” individual was born in the previous fall). 
The oldest individuals in the data set were 12 years old, although observations 
of individuals older than 6 years are sparse (Figure S2). 
In Mývatn, charr reach maturity between 4 and 5 years of age, 
and therefore we grouped the older individuals into a single reproductive age class, 
called “adults” or “age 4” for simplicity.

% ---------------------------------------------------------------------------------------
\subsection*{Demographic model} 
% ---------------------------------------------------------------------------------------

We characterized the demography of the charr popupaltion using an age structured model
with time-varying demographic rates.
The model projected the dynamics from one time step to the next as
%
\begin{equation} \label{eq:XPX}
    \mathbf{x}_t = \mathbf{P}_{t-1}~\mathbf{x}_{t-1}
\end{equation}
%
where $\mathbf{P}_{t}$ is the demographic projection matrix
and $\mathbf{x}_t$ is an age-structured vector of abundances at time $t$.
%
The projection matrix was defined as
%
\begin{equation} \label{eq:matrix}
\mathbf{P}_{i,t} = 
\left[
\begin{array}{cccccccc}
    0             & 0             & 0             & \rho_{t}     \\
    \phi_{1,t}    & 0             & 0             & 0            \\
    0             & \phi_{2,t}    & 0             & 0            \\
    0             & 0             & \phi_{3,t}    & \phi_{4,t}
    \end{array}
\right]
\end{equation}
%
where $\rho_{t}$ is per capita recruitment and $\phi_{1,t}$ is the survival probability
of age class $i$. 
The model assumes that only individuals of age four or greater reproduce,
and also allows individuals return to the adult age class.
%
Temporal variation in recruitment was modeled as a random walk on a log scale to ensure
that the values remained positive:
%
\begin{equation} \label{eq:rho}
\begin{aligned}
\text{log}\left(\rho_t\right) &= \text{log}\left(\rho_{t - 1}\right) + \epsilon_{\rho,t} \\
\epsilon_{\rho,t} &\sim \mathematical{N}\left(0, \sigma_{\rho} \right)
\end{aligned}
\end{equation}
%
where $\epsilon_{\rho,t}$ is the random walk step at time $t$.
The steps of the random walk followed a Gaussian distribution with mean 0 
and standard deviation $\epsilon_{\rho,t}$.
Temporal variation in survival probability for age class $i$ was modeled analagously 
on a logit-scale:
%
\begin{equation} \label{eq:phi}
\begin{aligned}
\text{logit}\left(\phi_{i, t}\right) &= \text{logit}\left(\phi_{i, t - 1}\right) +
                                          \epsilon_{\phi,i,t} \\
\epsilon_{\phi,i,t} &\sim \mathematical{N}\left(0, \sigma_{\phi} \right)
\end{aligned}
\end{equation}
\text{.}
%
Note that the random walks used to characterize the demographic parameters 
are not truly ``random'' as they are constrained by the data during model fitting.
Rather, they provided a convenient means 
of allowing the parameters to vary smoothly through time.

We paramterized the demographic model in a Bayesian framework, with likelihood
%
\begin{equation} \label{eq:likelihood}
\mathcal{L} = 
\displaystyle\prod_{s}
\displaystyle\prod_{i}
\displaystyle\prod_{t}
\text{Poisson}
    \left(
        y_{s,i,t}~|~\gamma_i~\times~x_{i,t}
    \right)
\end{equation}
%
where $y_{s,i,t}$ is the annual catch for station $s$ and age class $i$,
and $\gamma_i$ is the age-specific detection probability.
Estimating a separate detection probability for each age class allowed the model to account
for systematic differences in catch rates as a function of age.
Such differences are quite apparent in the data, 
as far too few first year individuals are captured to account for the abundance subsequent
ages, assuming a closed population. 
Our model is similar to an N-mixture model, 
however it does not include an explicit binomial detection process.
We opted for the simpler approach because we encountered substantial convergence problems
when attempting to fit a proper N-mixture model.

% ---------------------------------------------------------------------------------------
\section*{Results}
% ---------------------------------------------------------------------------------------

The full model provided a good visual fit to the station-level 
catch data (Figure \ref{fig:p_catch}),
with most of the variation in average catch being captured by the model.
The 90\% prediction intervals provided reasonable coverage relative to the observed data,
although the latter were slightly zero-inflated in some years.
While in principle it would be possible to account for such zero inflation,
it is unlikely that this would substantively alter the model inference.
In contrast to the full model,
the reduced model provided a poor fit to the data (Figure \ref{fig:p_catch_reduced}),
displaying damped oscillations that did not reflect the observed dynamics
(Figure \ref{fig:p_dens_reduced}).
This was corroborated the median posterior log-likelihood calculated
from equation \ref{eq:likelihood},
which was much higher for the full model (-4826) than for the reduced (-7314).
According to the full model, 
the detection probabilities increased with age,
with age 1 individuals having a catch probably an order of magnitude lower
than the other age classes (Table \ref{tab:param}).


M\'{y}vatn's arctic charr population fluctuated substantially over the 3-decade time series
(Figure \ref{fig:p_dens}), 
resulting from interannual variation in survival (Figure \ref{fig:p_surv}) 
and recruitment (Figure \ref{fig:p_rec}) 
as indicated by the random walk standard deviations (Table \ref{tab:param}).
The survival probabilities for all age classes increased through time,
although only the trend for adults was unambiguous (Table \ref{tab:gls}). 
In contrast, per capita recruitment declined through time, 
potentially counteracting the effects of increased adult survival.
Indeed, the geometric mean of the asymptotic growth rate across years was very close to one
(0.99; [0.970, 1.02]), indicating no long-term population change.
Furthermore, there was no clear trend in the population growth rate itself,
despite its large interannual fluctuations.
Together, these results show that while the arctic charr population was very dynamic
over the study period, it did not undergo meaningful directional change.




\section*{Discussion}

In this study, 
we used data from a multidecadal survey of arctic charr in Lake M\'{y}vatn
to quantify the population's demography and potential recovery 
following its collapse due to heavy exploitation 
\citep{gudbergsson2004}.
The survival probability of all age classes fluctuated substantially among years,
with only adults showing an unambiguous positive trend over the course of the study period.
In contrast, per capita recruitment clearly declined and experienced comparatively little
variation around this trend.
The countervailing changes in  per capita recruitment and adult survival resulted 
in no directional change in the total population size or population growth rate,
despite the rising number of adults.
Rather, the dynamics of  M\'{y}vatn's charr population 
were dominated by large interannual variability
and a shift towards a relatively older age distribution.
In and of itself, the increase in adult survival 
is a positive signal for the population's recovery,
and it has been taken as such by the local stakeholders 
who have called for relaxing harvest restrictions in recent years.
However, persistently low recruitment provides a cautionary note 
that should be taken into consideration in formulating a management strategy.

A crucial step for projecting the future dynamics of M\'{y}vatn’s charr population 
is identifying the underlying causes for the decline in per capita recruitment.
Food web interactions have been identified as an important source 
of fluctuations in other charr populations 
\citep{snorrason1992population, amundsen1994piscivory, jonsson2015freshwater}.
M\'{y}vatn is characterized by dramatic fluctuations in the abundance 
of primary food sources for juvenile charr, particularly benthic crustaceans 
and midges 
\citep{einarsson2004clad, gardarsson2004population, gudbergsson2004}.
Furthermore, there is substantial spatial heterogeneity in the abundance 
of these aquatic invertebrates \citep{bartrons2015spatial} 
which might disproportionately inhibit young juveniles 
that have more restricted mobility than the larger age classes. 
The large fluctuations in aquatic invertebrates are associated various consumer species 
in addition to charr, including sticklebacks and brown trout that could serve 
as competitors for young charr, and piscivorous waterfowl that could serve as predators 
\citep{einarsson2004myvatn}. 
In addition to biotic factors, 
temperature has received much attention as a driver of charr populations 
\citep{winfield2008arctic, elliott2010temperature, jonsson2015freshwater},
given their distribution restricted to arctic
and cold-temperature lakes \citep{klemetsen2003atlantic}
and the ubiquity anthropogenic climate change. 
While climate warming has not yet become an obvious ecological driver at M\'{y}vatn, 
it is nonetheless possible that temperature changes have adversely affected recruitment 
in M\'{y}vatn’s charr population as has been seen in other Icelandic lakes 
\citep{malmquist2009salmonid}. 

This study provides a demographic assessment of a population of arctic charr in a single lake,
but it reinforces themes that have broad and increasing interest in applied ecology.
Harvest-induced shifts in age structure have been documented in other fisheries,
typically resulting in ``truncation'' of the oldest age classes 
that are typically the targets of harvest efforts
\citep{hsieh2010fishing}.
Suppression of adult abundance is expected to have deleterious effects on populations,
with the corollary that relaxation of harvest should allow populations 
to recover following overexploitation.
However, this will only be true if juvenile recruitment is sufficient 
to sustaining the population's recovery.
Recruitment in fish populations 
has long been recognized as highly variable and difficult to predict
owing to the interplay of numerous biotic and abiotic factors 
\citep{dixon1999episodic, houde2008emerging, ludsin2014physical}.
This poses a particular challenge for management efforts, 
as complex suites of ecological factors are both difficult to understand 
and difficult to regulate, 
particularly in comparison to a direct anthropogenic driver such as harvest
\citep{beamish1999taking, link2002does}.
The extent to which this is true for M\'{y}vatn's arctic charr is currently unknown.
Nonetheless, 
the countervailing trends in survival and recruitment 
in the wake of alleviated of harvesting pressure
underscore the potential for heterogeneous demographic responses to management efforts
due to the complex ecological context in which such efforts take place.







% ---------------------------------------------------------------------------------------
% ---------------------------------------------------------------------------------------
% References, appendix, figures, etc.
% ---------------------------------------------------------------------------------------
% ---------------------------------------------------------------------------------------

\bibliographystyle{ecology.bst}
\clearpage

\bibliography{refs.bib}

% ---------------------------------------------------------------------------------------
\clearpage
\begin{table}
\caption{\label{tab:param}
Posterior summaries of the parameters from the demographic model.
Catch rates are dimensionless and constrained between 0 and 1,
while the random walk SD are on the scale of either logit-survival or log-recruitment
and constrained to be >0.
Uncertainty intervals are based on 68\% quantiles, 
matching the nominal coverage of standard errors.
}
\setlength{\tabcolsep}{12pt}
\begin{tabular}{llc}
\toprule
Model parameter        &                      & Posterior median [\text{UI}_{68\%}] \\
\cmidrule{1-3}
catch rate             & age 1                & 0.001 [0.0006, 0.002]               \\
&                        age 2                & 0.08 [0.05, 0.13]                   \\
&                        age 3                & 0.48 [0.32, 0.67]                   \\
&                        age 4+               & 0.66 [0.48, 0.83]                   \\
random walk SD         & survival probability & 2.03 [1.79, 2.30]                   \\
&                        recruitment          & 1.45 [1.25, 1.70]                   \\
\bottomrule
\end{tabular}
\end{table}
\clearpage
% ---------------------------------------------------------------------------------------

% ---------------------------------------------------------------------------------------
\clearpage
\begin{table}
\caption{\label{tab:gls}
Coefficients and standard errors (SE) from the GLS models 
quantifying linear trends in the demographic rates through time.
The models were fit on either a log (recruitment; population growth rate)
or logit (survival probability) scale,
and all response and predictor variables were z-scored prior to model fitting.
Therefore, the coefficients can be compared as effect sizes 
across all of the response variables.
The autoregressive (AR) coefficients were included to account for temporal autocorrelation
when estimating the year trends;
0 indicates no autocorrelation, 
while |1| indicates strong (positive or negative) autocorrelation 
and statistical non-stationarity.
}
\setlength{\tabcolsep}{12pt}
\begin{tabular}{llccc}
\toprule
Response variable      &        & Intercept (\pm~\text{SE}) & Year slope (\pm~\text{SE}) & AR
                                                                                coefficient \\
\cmidrule{1-5}
survival probability   & age 1  & ~0.01  \pm~0.23           & ~0.26 \pm~0.23        & ~0.26 \\
&                        age 2  & ~0.00  \pm~0.15           & ~0.13 \pm~0.15        & -0.18 \\
&                        age 3  & ~0.04  \pm~0.54           & ~0.25 \pm~0.44        & ~0.80 \\
&                        age 4+ & -0.02 \pm~0.26            & ~0.55 \pm~0.25        & ~0.53 \\
recruitment            &        & -0.06 \pm~0.29            & -0.71 \pm~0.27        & ~0.56 \\
population growth rate &        & ~0.00  \pm~0.15           & ~0.08 \pm~0.15        & -0.21 \\
\bottomrule
\end{tabular}
\end{table}
\clearpage
% ---------------------------------------------------------------------------------------


% ---------------------------------------------------------------------------------------
\clearpage
\begin{figure}
\centering
\includegraphics{../analysis/p_catch.pdf}
\caption{\label{fig:p_catch}
Station-level catch (points) and 90\% posterior prediction intervals (shading) 
from the demographic model.
The prediction intervals include stochasticity arising from the Poisson sampling process,
and therefore represent the predicted distribution of catch according to the model. 
The solid lines are medians of the the prediction distributions and quantify the expected
catch; note that this is not the same quantity as the estimated population density, 
shown in Figure \ref{fig:p_dens}.
The y-axis is log+1 transformed to accommodate zeros.
}
\end{figure}
\clearpage
% ---------------------------------------------------------------------------------------

% ---------------------------------------------------------------------------------------
\clearpage
\begin{figure}
\centering
\includegraphics{../analysis/p_dens.pdf}
\caption{\label{fig:p_dens}
Lake-wide population density (solid line) inferred from the demographic model.
Shading depicts 68\% uncertainty intervals, 
matching the nominal coverage of standard errors.
The y-axis is log transformed.
}
\end{figure}
\clearpage
% ---------------------------------------------------------------------------------------

% ---------------------------------------------------------------------------------------
\clearpage
\begin{figure}
\centering
\includegraphics{../analysis/p_lam.pdf}
\caption{\label{fig:p_lam}
Asymptotic population growth rate (solid blue line) inferred from the demographic model.
Shading depicts 68\% uncertainty intervals, 
matching the nominal coverage of standard errors.
The dashed horizontal line indiates a growth rate of 1,
which corresponds to no change in the population size from one time step to the next.
The solid black line represents the fitted values from the GLS model
fit on a log-scale and then back-transformed 
to match the scale of the population growth rate.
Note however, that the y-axis of the figure is log-transformed, 
thereby preserving the linearity of the data and model fit.
}
\end{figure}
\clearpage
% ---------------------------------------------------------------------------------------

% ---------------------------------------------------------------------------------------
\clearpage
\begin{figure}
\centering
\includegraphics{../analysis/p_surv.pdf}
\caption{\label{fig:p_surv}
Logit-survival probability (solid colored lines) inferred from the demographic model.
Shading depicts 68\% uncertainty intervals, 
matching the nominal coverage of standard errors.
The solid black line represents the fitted values from the GLS models
fit separately for each age class.
Note that logit-survival probability ranging from -5 to 5 
corresponds to actual survival probability ranging from 0.007 to 0.993.
}
\end{figure}
\clearpage
% ---------------------------------------------------------------------------------------

% ---------------------------------------------------------------------------------------
\clearpage
\begin{figure}
\centering
\includegraphics{../analysis/p_rec.pdf}
\caption{\label{fig:p_rec}
Log-recruitment $\text{capita}^{-1}$ (solid blue line) inferred from the demographic model.
Shading depicts 68\% uncertainty intervals, 
matching the nominal coverage of standard errors.
The solid black line represents the fitted values from the GLS models.
Note that log-recruitment ranging from -1 to 8 
corresponds to actual log-recruitment ranging from 0.4 to 3000.
}
\end{figure}
\clearpage
% ---------------------------------------------------------------------------------------






% ---------------------------------------------------------------------------------------
% ---------------------------------------------------------------------------------------
% Supplemental figures
% ---------------------------------------------------------------------------------------
% ---------------------------------------------------------------------------------------

\renewcommand{\thefigure}{S\arabic{figure}}
\renewcommand{\theequation}{S\arabic{equation}}
\renewcommand{\thetable}{S\arabic{table}}
\setcounter{equation}{0}
\setcounter{figure}{0}
\setcounter{table}{0}

% ---------------------------------------------------------------------------------------
\begin{figure}
\centering
\includegraphics{../analysis/p_catch_reduced.pdf}
\caption{\label{fig:p_catch_reduced}
Station-level catch (points) and 90\% posterior prediction intervals (shading) 
from the reduced version of the demographic model 
with survival and recruitment fixed through time.
The prediction intervals include stochasticity arising from the Poisson sampling process,
and therefore represent the predicted distribution of catch according to the model. 
The solid lines are medians of the the prediction distributions and quantify the expected
catch; note that this is not the same quantity as the estimated population density, 
shown in Figure \ref{fig:p_dens_reduced}.
The y-axis is log+1 transformed to accommodate zeros.
}
\end{figure}
\clearpage
% ---------------------------------------------------------------------------------------

% ---------------------------------------------------------------------------------------
\clearpage
\begin{figure}
\centering
\includegraphics{../analysis/p_dens_reduced.pdf}
\caption{\label{fig:p_dens_reduced}
Lake-wide population density (solid line) inferred from the 
reduced version of the demographic model 
with survival and recruitment fixed through time.
Shading depicts 68\% uncertainty intervals, 
matching the nominal coverage of standard errors.
The y-axis is log transformed.
}
\end{figure}
\clearpage
% ---------------------------------------------------------------------------------------

\end{document}

