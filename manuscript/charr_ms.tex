% ---------------------------------------------------------------------------------------
% ---------------------------------------------------------------------------------------
% Preliminaries
% ---------------------------------------------------------------------------------------
% ---------------------------------------------------------------------------------------

\documentclass[11pt]{article}
\usepackage[letterpaper, margin=1in]{geometry}
\usepackage{newtxtext,newtxmath}
\usepackage[math-style=ISO]{unicode-math}
\usepackage{fullpage}
\usepackage[authoryear,sectionbib]{natbib}
\linespread{1.7}
\usepackage[utf8]{inputenc}
\usepackage{lineno}
\usepackage{titlesec}
\titleformat{\section}[block]{\Large\bfseries\filcenter}{\thesection}{1em}{}
\titleformat{\subsection}[block]{\Large\itshape\filcenter}{\thesubsection}{1em}{}
\titleformat{\subsubsection}[block]{\large\itshape}{\thesubsubsection}{1em}{}
\titleformat{\paragraph}[runin]{\itshape}{\theparagraph}{1em}{}[. ]\renewcommand{\refname}
  {Literature Cited}
\DeclareTextSymbolDefault{\dh}{T1} % for Icelandic ð symbol:
\usepackage{graphicx} % for figures
\usepackage{booktabs} % for tables
\usepackage{amsmath} % for split math environment
\usepackage{enumitem}




% ---------------------------------------------------------------------------------------
% ---------------------------------------------------------------------------------------
% Title page
% ---------------------------------------------------------------------------------------
% ---------------------------------------------------------------------------------------


\title{Opposing trends in survival and recruitment slow the recovery 
        of an historically overexploited fishery}

\author{
Joseph S. Phillips$^{1,2, \dagger}$ \\
Gu{\dh}ni Gu{\dh}bergsson$^{3}$ \\
Anthony R. Ives$^{2}$
}

\date{}

\begin{document}

\raggedright
\setlength\parindent{0.25in}

\maketitle


\noindent{} 1. Department of Aquaculture and Fish Biology, 
H\'{o}lar University, Skagafj\"{o}r{\dh}ur, Iceland

\noindent{} 2. Department of Integrative Biology, 
University of Wisconsin, Madison, Wisconsin, USA

\noindent{} 3. Marine and Freshwater Research Institute, Hafnafj\"{o}r{\dh}ur, Iceland

\noindent{} $\dagger$ E-mail: joseph@holar.is



\bigskip

Running head: {Arctic charr demography}

\textit{Keywords}: {age structure;
                    demography; 
                    Lake M\'{y}vatn; 
                    population dynamics; 
                    \emph{Salvelinus alpinus}}

\clearpage





% ---------------------------------------------------------------------------------------
% ---------------------------------------------------------------------------------------
% Abstract
% ---------------------------------------------------------------------------------------
% ---------------------------------------------------------------------------------------

\linenumbers{}

\section*{Abstract} \label{abstract}

\begin{enumerate}[label=\arabic*.]
\item
Quantifying temporal variation in demographic rates is a central goal of population ecology
in both basic and applied settings.

\item
In this study, we analyzed a multidecadal age-structured time series 
of Arctic charr (\emph{Salvelinus alpinus}) abundance  in Lake M\'{y}vatn, Iceland, 
to infer the time-varying demographic response of the population 
to reduced harvest in the wake of the fishery's collapse.

\item
Our analysis shows that while survival probability of adults increased following the alleviation
of harvesting pressure, 
per capita recruitment consistently declined over most of the study period,
until the final three years when it began to increase.
The countervailing demographic trends resulted 
in only limited directional change in the total population size and population growth rate.
Rather, the population dynamics were dominated by large interannual variability
and a shift towards an older age distribution.

\item
Our results are indicative of a slow recovery of the population after its collapse,
despite the rising number of adults following relaxed harvest.
This underscores the potential for heterogeneous demographic responses to management efforts
due to the complex ecological context in which such efforts take place.

\end{enumerate}

\bigskip

\clearpage



% ---------------------------------------------------------------------------------------
% ---------------------------------------------------------------------------------------
% Introduction
% ---------------------------------------------------------------------------------------
% ---------------------------------------------------------------------------------------

\section*{Introduction}

Quantifying temporal variation in demographic rates is a central goal of population ecology,
because this underpins efforts to characterize both exogenous and 
endogenous drivers of population dynamics 
\citep{twombly1994comparative, zeng1998, koons2016life}.
However, this endeavor is challenging, 
as the characterization of directional trends in demographic rates requires
data spanning many years and 
often relies on intensive mark-recapture (and related) approaches for statistical inference
\citep[e.g.,][]{forcada2008life, hunter2010climate}.
Both of these challenges are amplified in populations subject to large interannual variation
in demographic rates, 
which are often the targets of both basic and applied interest
\citep[e.g.,][]{white2007irruptive}.
Consequently, 
there is a need for additional studies that 
explicitly quantify temporal variation in demographic rates 
and the resulting population dynamics, 
particularly for populations that have not been the subjects 
of intensive mark-recapture-style campaigns.

In this study, we analyzed a multidecadal age-structured time series 
of Arctic charr (\emph{Salvelinus alpinus}) abundance in Lake M\'{y}vatn, Iceland, 
to infer the time-varying demographic response of the population 
to reduced harvest in the wake of the fishery's collapse.
Arctic charr are salmonids with a circumpolar distribution
and are the targets of 
numerous commercial fisheries \citep{klemetsen2003atlantic}.
Moreover,
Arctic charr have important effects in many arctic 
and boreal freshwater food webs owing to their roles as top consumers
\citep{jeppesen2001fish, klemetsen2003atlantic}.
In M\'{y}vatn, the charr sustained a large commercial fishery throughout much of the 
twentieth century that was subject to exploitation rates upwards of 80\%,
proceeding the fishery's collapse by the late 1980s \citep{gudbergsson2004}.
This collapse led to catch restrictions imposed in the early 2000s.
A portion of the lake was also dredged during this period,
which may have indirectly contributed to the decline of the fishery
through depressed prey availability \citep{einarsson2004myvatn}.
A monitoring program was instituted in 1986 to track the population's
dynamics in the wake of these ecological perturbations.
This monitoring was conducted by one of the authors (GG) 
using the same methods and sampling locations within the lake 
for the entire time series,
providing a valuable opportunity to quantify 
long-term demographic changes in the population.

Using these monitoring data, 
we parameterized an age-structured demographic model with time-varying recruitment 
and survival to characterize the dynamics of the population.
We modeled temporal variation in the demographic rates as random walks, 
an approach that had previously been applied 
to infer population growth rates from non-structured abundance 
\citep{zeng1998, ives2012detecting}
and for age-specific mortality rates inferred from fisheries stock assessments 
\citep{nielsen2014estimation}. 
This method takes advantage of the entire time series for estimating the parameters 
while allowing them to vary smoothly through time. 
Furthermore, the model is able to characterize a range of dynamics including those
arising from negative density-dependence and environmental perturbations
\citep{ives2012detecting}.
The latter may be particularly important in the case of the M\'{y}vatn's charr,
because the lake is subject to large fluctuations 
in primary \citep{phillips2020time, mccormick2021shifts} 
and secondary \citep{einarsson2004clad, gardarsson2004population}
production that may cascade up to the charr population.
Accordingly, the purpose of our analysis is to 
(i) characterize interannual variability in survival and recruitment 
in M\'{y}vatn's charr population, 
and (ii) determine whether directional trends in these demographic rates have
resulted in the recovery of the population in the wake of its collapse. 




% ---------------------------------------------------------------------------------------
% ---------------------------------------------------------------------------------------
% Methods
% ---------------------------------------------------------------------------------------
% ---------------------------------------------------------------------------------------

% ---------------------------------------------------------------------------------------
\section*{Methods} 
% ---------------------------------------------------------------------------------------

% ---------------------------------------------------------------------------------------
\subsection*{Study system} 
% ---------------------------------------------------------------------------------------

M\'{y}vatn is located in northeastern Iceland (65°40’N 17°00’W) 
and has a tundra-subarctic climate \citep{bjornsson2004climate, einarsson2004myvatn}. 
The lake spans $37 \text{km}^2$, 
divided into north (Ytrifloi; $8 \text{km}^2$) 
and south (Sy{\dh}rifloi; $29 \text{km}^2$) basins 
connected by a narrow channel \citep{einarsson2004myvatn}. 
M\'{y}vatn is shallow (south basin mean depth = 2.3m) and 
fed by nutrient-rich springs that sustain high benthic primary and secondary production.
The latter comprises large but temporally variable populations of benthic invertebrates 
such as midges and cladocerans \citep{einarsson2004clad, gardarsson2004population}.
The benthic invertebrates are in turn 
an important food source for M\'{y}vatn’s vertebrate populations, 
including Arctic charr, threespine stickleback (\emph{Gasterosteus aculeatus}), 
brown trout (\emph{Salmo trutta}), and waterfowl \citep{einarsson2004myvatn}. 
The north basin was historically dredged (1967--2003), 
which likely affected many populations in the lake, 
although such effects are poorly understood \citep{einarsson2004myvatn}.

While some Arctic charr populations are anadromous, 
M\'{y}vatn’s population appears to reside strictly within the lake; 
there is major outflow that connects the lake to the Greenland Sea,
but natural barriers prevent individuals from entering the lake from downriver
\citep{gudbergsson2004}. 
There are two charr morphs within M\'{y}vatn: pelagic (size at maturation of 35--50cm) 
and small benthic (20--25cm). 
However, the small benthic morph is restricted to a small 
and relatively isolated part of the south basin fed by cold springs 
\citep{gudbergsson2004}; 
therefore, the pelagic morph is the focus of this study. 
Gut content data reveal a diverse diet, including midges, snails, clams, zooplankton, 
benthic crustaceans, and sticklebacks as the main prey species. 
The large-bodied cladoceran \emph{Eurycercus lamellatus} 
is thought to be a particularly important prey item 
\citep{gudbergsson2004}. 
Some charr populations are cannibalistic \citep{klemetsen2003atlantic}, 
especially in the absence of other large-bodied prey (e.g., fish).
However, the charr in M\'{y}vatn appear to lack cannibalism based on gut content data
\citep{gudbergsson2004}. 

% ---------------------------------------------------------------------------------------
\subsection*{Study design} 
% ---------------------------------------------------------------------------------------

Systematic surveys of M\'{y}vatn’s Arctic charr have been conducted 
with monofilament gill nets every year from 1986--2020 by a single researcher (GG) 
using a consistent methodology. 
A series of mesh sizes (16.5, 18.5, 21.5, 25, 30, 35, 40, 46, and 50mm) 
was used to capture individuals over a large size range;
smaller nets to target young-of-the-year were not used to avoid stickleback by-catch.
The surveys largely took place after the period of most dramatic population decline 
\citep{gudbergsson2004} 
and were intended to monitor the recovery of the population following its collapse. 
Twelve survey stations were sampled from around the lake,
ten of which were located in the south basin;
two of the south basin stations were aggregated for the purpose this analysis
due to their close proximity to each other. 
Captured individuals were aged either by otoliths directly or 
with estimates based on length.
The surveys were conducted in fall (late August through September) 
of every year before spawning and also in June in a subset of years. 
For this analysis we used only the August/September data. 
Therefore, in our age classifications an individual of age ``x'' is an individual 
that survived to that age (e.g., an ``age 1'' individual hatched in the previous fall). 
The oldest individuals in the data set were 12 years old, although observations 
of individuals older than 6 years are sparse
(Supporting Information: Figure S1). 
In M\'{y}vatn, charr reach maturity at 4--5 years of age and 
typically spawn multiple times after maturity, 
and therefore we grouped the older individuals into a single reproductive age class
denoted ``age 4+''.

% ---------------------------------------------------------------------------------------
\subsection*{Data analysis} 
% ---------------------------------------------------------------------------------------

We characterized the demography of the charr population 
using an age-structured model \citep{caswell2001matrix}
with time-varying demographic rates 
\citep{zeng1998, ives2012detecting, nielsen2014estimation}.
The model projected the dynamics from one time step to the next as
%
\begin{equation} \label{eq:XPX}
    \mathbf{x}_t = \mathbf{P}_{t-1}~\mathbf{x}_{t-1}
\end{equation}
%
where $\mathbf{P}_{t}$ is a demographic projection matrix,
and $\mathbf{x}_t$ is an age-structured vector of population densities at time $t$.
%
The projection matrix was defined as
%
\begin{equation} \label{eq:matrix}
\mathbf{P}_{t} = 
\left[
\begin{array}{cccccccc}
    0             & 0             & 0             & \rho_{t}     \\
    \phi_{1,t}    & 0             & 0             & 0            \\
    0             & \phi_{2,t}    & 0             & 0            \\
    0             & 0             & \phi_{3,t}    & \phi_{4,t}
    \end{array}
\right]
\end{equation}
%
where $\rho_{t}$ is per capita recruitment 
and $\phi_{1,t}$ is the survival probability of age class $i$. 
The model includes only four age classes,
and individuals surviving beyond age 4 return to the age 4+ class.

Temporal variation in recruitment was modeled as a random walk on a log scale to ensure
that the values remained positive:
%
\begin{equation} \label{eq:rho}
\begin{aligned}
\rho_t &= \text{exp}\left(\chi_t\right) \\
\chi_t &= \chi_{t-1} + \epsilon_{\rho,t} \\
\epsilon_{\rho,t} &\sim \text{Gaussian}\left(0, \sigma_{\rho} \right)
\end{aligned}
\end{equation}
%
where $\epsilon_{\rho,t}$ is the random walk step at time $t$.
The steps of the random walk followed a Gaussian distribution with mean of 0 
and standard deviation (SD) of $\epsilon_{\rho,t}$.
Temporal variation in survival probability for age class $i$ was modeled analogously 
on a logit scale:
%
\begin{equation} \label{eq:phi}
\begin{aligned}
\phi_{i, t} &= \text{logit}^{-1}\left(\gamma_{i, t}\right) \\
\gamma_{i, t} &= \gamma_{i, t - 1} + \epsilon_{\phi,i,t} \\
\epsilon_{\phi,i,t} &\sim \text{Gaussian}\left(0, \sigma_{\phi} \right)\text{.}
\end{aligned}
\end{equation}
%
with the same random walk SD used for all age classes 
to reduce the number of parameters fit by the model.
Note that the ``random walks'' were constrained by the data during model fitting,
providing a convenient means of allowing the parameters to vary smoothly through time
in accordance with the data
\citep{zeng1998, ives2012detecting}.
The random walk SDs for recruitment and survival 
characterize the degree of temporal variation in the respective processes. 

We fit the model in a Bayesian framework with a zero-inflated Poisson likelihood:
%
\begin{equation} \label{eq:likelihood}
\mathcal{L} = 
\displaystyle\prod_{s}
\displaystyle\prod_{i}
\displaystyle\prod_{t}
\begin{cases} 
      \theta + (1-\theta) \times \text{Poisson}
        \left(
            0~|~\kappa_i~\times~x_{i,t}
        \right) & \text{if}~y_{s,i,t} = 0 \\
      (1-\theta) \times \text{Poisson}
        \left(
            y_{s,i,t}~|~\kappa_i~\times~x_{i,t}
        \right) & \text{if}~y_{s,i,t} > 0
   \end{cases}
\end{equation}
%
where $y_{s,i,t}$ is the annual survey catch for station $s$ and age class $i$,
$\kappa_i$ is the age-specific sampling efficiency (constrained between 0 and 1)
and $\theta$ is the probability of zero catch from a Bernoulli sampling processes
(``zero-inflation rate'').
The likelihood implies that the station-level abundance follows a zero-inflated 
Poisson distribution with a rate parameter equal to the lake-wide mean population density 
scaled by the sampling efficiency for the corresponding age class.
We assumed that the lake-wide population was well-mixed across years,
which is supported by a mark-recapture study showing that individuals tagged
at spawning grounds were recovered in all parts of the lake \citep{gudbergsson1991}.
Moreover, while we did not attempt to account for systematic differences between stations,
no such differences were consistently apparent from visual inspection of the data 
(Supporting Information: Figure S2).
Estimating a separate sampling efficiency for each age class allowed the model to account
for systematic differences in sampling efficiency as a function of age.
Such differences were visually apparent in the data, 
as far too few first-year individuals were captured 
to account for the abundance of subsequent ages assuming a closed population
(Supporting Information: Figure S1). 
Note that zero-inflation is independent of age-specific variation in sampling efficiency
and is best understood as the consequence of processes unrelated 
to age- or size-specific sampling efficiency.

During model fitting, we divided $x_{i,t}$ by 100 
to improve computational efficiency by avoiding excessively largely values;
we then backscaled the estimates for $x_{i,t}$ in reporting the results.
We used exponential priors with rate 1/50 for the initial scaled population densities for
each age class, 
Gaussian priors with mean 0 and SD of either 5 or 1 for 
initial log-recruitment or logit-survival (respectively), 
Gamma priors with shape and scale both set to 1.5 for the random walk SDs,
and Beta priors with shape parameters both set to 2 for the sampling efficiency
and zero-inflation rate.
The overall scales of these priors were weakly informative relative to the 
scales of the corresponding parameters.
The Gamma priors were unimodal, concave-down near zero, and had zero density at zero,
allowing the resulting posteriors to be arbitrarily close to zero 
while not being drawn towards zero by the prior.
The Beta priors were unimodal and concave-down, with a mode at 0.5.

To assess temporal trends in the demographic rates, 
we used generalized least squares  (GLS) to fit lag-1 autoregressive models
with linear year effects.
Recruitment and survival probabilities were fit on their corresponding link scales 
(either log or logit),
and survival probabilities were fit for each age class separately.
We used a similar approach to estimate trends in the asymptotic population growth rate
(log scale), 
calculated as the leading eigenvalue of projection matrix $\mathbf{P}_{t}$.
We performed this analysis on the asymptotic (as opposed to realized) growth rate 
because we were primarily interested in the effects 
of the underlying demographic rates per se; 
the realized growth rate would also have included transient fluctuations due to
non-equilibrium age structure \citep{caswell2001matrix}.
For all GLS models, the response variables and year were z-transformed
(subtracted mean and divided by SD) so that coefficients
could be interpreted as effects sizes;
fitted values were then back-scaled for the figures.

The demographic model was fit using the statistical language Stan \citep{carpenter2017stan},
with 4 chains, 2000 iterations, maximum tree depth set to 14, and adapt delta set to 0.95.
Convergence and quality of MCMC sampling were assessed using the diagnostics provided by Stan, 
including ``Rhat'', the number of divergences, and the effective sample size.
We used posterior medians as point estimates 
and the bounds of 68\% posterior quantiles as uncertainty intervals 
(hereafter $\text{UI}_{68\%}$), 
matching the nominal coverage of standard errors.
To gauge the degree to which the data were overdispersed relative 
to the model expectations, 
we simulated 90\% prediction intervals from equation \ref{eq:likelihood} 
and compared the coverage of these intervals to the data.
To assess the extent to which the data contained statistically meaningful information 
regarding temporal variation in the demographic rates,
we compared the fit of the full model to a reduced model with the demographic
rates fixed through time.
We used Stan via the \texttt{rstan} package \citep{stan2020rstan},
performed GLS with the \texttt{gls} function 
from the \texttt{nlme} package \citep{pinheiro2020nlme}, 
and calculated the asymptotic growth rate with 
the \texttt{demogR} package \citep{holland2007demogR}.
All analysis were conducted in R 4.0.3 \citep{r2020}. 





% ---------------------------------------------------------------------------------------
% ---------------------------------------------------------------------------------------
% Results
% ---------------------------------------------------------------------------------------
% ---------------------------------------------------------------------------------------

% ---------------------------------------------------------------------------------------
\section*{Results}
% ---------------------------------------------------------------------------------------

The full model provided a good visual fit to the station-level 
catch data (Figure \ref{fig:p_catch}a),
with most of the temporal variation in lake-wide catch being well characterized by the model.
Moreover, 
the 90\% prediction intervals provided reasonable coverage relative to the observed data.
In contrast to the full model,
the reduced model without time-varying parameters
provided a poor fit to the data (Figure \ref{fig:p_catch}b),
displaying damped oscillations that did not reflect the observed dynamics
(Supporting Information: Figure S3).
This was corroborated by the median posterior log-likelihood calculated
from equation \ref{eq:likelihood},
which was much higher for the full model (-4065) than for the reduced (-6075).
According to the full model, 
the sampling efficiencies increased with age;
the sampling efficiency for age 1 individuals was around two orders of magnitude lower
than for the other age classes (Table \ref{tab:param}). 
This is consistent with the expectation that age 1 individuals 
are generally too small to be captured by the gill nets used in the surveys.

M\'{y}vatn's Arctic charr population fluctuated substantially over the 3-decade time series
(Figure \ref{fig:p_dens}), 
with the asymptotic population growth rate varying across an order of magnitude 
(Figure \ref{fig:p_lam}).
This variation was underpinned by substantial variation in survival 
(Figure \ref{fig:p_surv}) 
and recruitment (Figure \ref{fig:p_rec}),
both of which had random walk SDs 
with posterior densities concentrated away from zero (Table \ref{tab:param}).
The survival probabilities for all age classes increased through time,
although only the trend for age 4+ was statistically unambiguous 
(Figure \ref{fig:p_surv}; Table \ref{tab:gls}).
Elevated age 4+ survival was associated with a steady increase in the survey catch of
age 4+ individuals from 2005 onward (Figure \ref{fig:p_catch}). 
In contrast, per capita recruitment declined over most of the time series, 
consistent with declining age 1 survey catch
(Figure \ref{fig:p_rec}; Table \ref{tab:gls}).
However, per capita recruitment increased in the most recent three years,
again corresponding with the  age 1 survey catch.
The positive effect of increased survival was potentially negated by the decline in recruitment.
Indeed, the geometric mean of the population growth rate across years was very close to one
(1.021 [0.997, 1.046]), indicating no long-term population change.
Furthermore, the trend in the population growth rate itself was only slightly positive 
and statistically ambiguous,
especially when judged against its large interannual fluctuations
(Figure \ref{fig:p_lam}; Table \ref{tab:gls}).
The AR coefficients were moderately positive for most of the demographic parameters,
except for age 2 survival and the population growth rate 
for which the AR coefficients were somewhat negative.
These negative coefficients potentially indicate overcompensatory dynamics,
in which high values in one year predict low values in the next.
Nonetheless, this inference is tentative given their low magnitudes 
and absence of uncertainty estimates.
Overall, the GLS results show that while the size of the Arctic charr population 
was variable over the study period, 
it did not undergo meaningful directional change.





% ---------------------------------------------------------------------------------------
% ---------------------------------------------------------------------------------------
% Discussion
% ---------------------------------------------------------------------------------------
% ---------------------------------------------------------------------------------------


\section*{Discussion}

In this study, 
we used data from a multidecadal survey of Arctic charr in Lake M\'{y}vatn
to quantify demographic changes and potential recovery of the population
following its collapse.
The survival probability of all age classes fluctuated substantially among years,
with only adults showing an unambiguous positive trend over the course of the study period.
In contrast, per capita recruitment declined substantially and experienced comparatively little
variation around this trend until an increase over the most recent three years.
The countervailing changes in  per capita recruitment and adult survival resulted 
essentially no directional change in overall population size 
and only a very limited increase in the population growth rate,
despite the rising number of adults.
Rather, the dynamics of  M\'{y}vatn's charr population 
were dominated by large interannual variability
and a shift towards a relatively older age distribution.
The somewhat elevated recruitment in recent years, alongside relatively high adult survival,
does suggest that the population may in fact be recovering.
However, any such recovery appears to be in its early stages and 
has taken several decades to manifest.
Therefore, recent calls from local stakeholders to relax harvest restrictions 
in response to increased adult survival should be tempered by caution
until the population's recovery has been more clearly established.

A crucial step for projecting the future dynamics of M\'{y}vatn’s charr population 
is identifying the underlying causes for the decline in per capita recruitment.
Food-web interactions have been identified as an important source 
of fluctuations in other charr populations 
\citep{snorrason1992population, amundsen1994piscivory, jonsson2015freshwater}.
M\'{y}vatn is characterized by dramatic fluctuations in the abundance 
of primary food sources for juvenile charr, particularly benthic crustaceans 
and midges 
\citep{einarsson2004clad, gardarsson2004population, gudbergsson2004}.
Furthermore, there is substantial spatial heterogeneity in the abundance 
of these aquatic invertebrates \citep{bartrons2015spatial}, 
which might disproportionately inhibit young juveniles 
that have more restricted mobility than the larger age classes. 
The large fluctuations in aquatic invertebrates are associated 
with fluctuations in populations of other consumer species in addition to charr, 
including threespine stickleback and brown trout that could serve 
as competitors with young charr, and piscivorous waterfowl that could serve as predators 
\citep{einarsson2004myvatn}. 
In addition to biotic factors, 
temperature has received much attention as a driver of charr populations 
\citep{winfield2008arctic, elliott2010temperature, gerdeaux2011does, jonsson2015freshwater},
given their distribution restricted to arctic
and cold-temperature lakes \citep{klemetsen2003atlantic},
and the ubiquity anthropogenic climate change. 
While climate warming has not yet become an obvious ecological driver at M\'{y}vatn, 
it is nonetheless possible that temperature changes have adversely affected recruitment 
in M\'{y}vatn’s charr population as has been seen in other lakes in Iceland and Scandanavia
\citep{malmquist2009salmonid, svenning2021temporal}

This study provides a demographic assessment of a population of Arctic charr in a single lake,
but it reinforces themes that have broad and increasing interest in applied ecology.
Harvest-induced shifts in age structure have been documented in other fisheries,
typically resulting in ``truncation'' of the oldest age classes 
that are typically the targets of harvest efforts
\citep{hsieh2010fishing}.
Suppression of adult abundance is expected to have deleterious effects on populations,
with the corollary that relaxation of harvest should allow populations 
to recover following overexploitation.
However, this will only be true if juvenile recruitment is sufficient 
to sustain the population's recovery.
Recruitment in fish populations 
has long been recognized as highly variable and difficult to predict
owing to the interplay of numerous biotic and abiotic factors 
\citep{dixon1999episodic, houde2008emerging, ludsin2014physical}.
This poses a particular challenge for management efforts, 
as complex suites of ecological factors are both difficult to understand 
and difficult to regulate, 
particularly in comparison to a direct anthropogenic driver such as harvest
\citep{beamish1999taking, link2002does}.
The extent to which this is true for M\'{y}vatn's Arctic charr is currently unknown.
Nonetheless, 
the countervailing trends in survival and recruitment 
in the wake of alleviated of harvesting pressure
underscore the potential for heterogeneous demographic responses to management efforts
due to the complex ecological context in which such efforts take place.





% ---------------------------------------------------------------------------------------
% ---------------------------------------------------------------------------------------
% References, appendix, figures, etc.
% ---------------------------------------------------------------------------------------
% ---------------------------------------------------------------------------------------

\section*{Acknowledgments} 

The data used in this manuscript were collected during routine sampling 
of M\'{y}vatn’s charr population supported by 
the Icelandic Marine and Freshwater Research Institute 
and the M\'{y}vatn Research Station. 
Further support came from NSF LTREB DEB-1556208 to ARI and 
NSF Graduate Research Fellowship (DGE-1256259) supporting JSP.
GG conducted the surveys of M\'{y}vatn’s Arctic charr population,
with contributions from staff of the Marine and Freshwater Research Institute. 
JSP conducted the analyses, 
with input from ARI. 
JSP wrote the first draft of the paper, 
with substantial contributions from all authors to subsequent drafts.
We thank \'{A}rni Einarsson, Camille Leblanc, and Bjarni Krist\'{o}fer Kristj\'{a}nsson
for their constructive feedback on the manuscript.

\section*{Data and code availability}

Data and code will be made available on FigShare upon acceptance.

\clearpage

\bibliographystyle{ecology.bst}

\bibliography{refs.bib}

% ---------------------------------------------------------------------------------------
\clearpage
\begin{table}
\caption{\label{tab:param}
Posterior summaries of the parameters from the demographic model.
Catch rates are dimensionless and constrained between 0 and 1,
while the random walk SD are on the scale of either logit-survival or log-recruitment
and constrained to be positive.
Uncertainty intervals are based on 68\% quantiles, 
matching the nominal coverage of standard errors.
Note that initial values for the random walks and 
age-specific population density are not shown in this table
but can be seen in the corresponding figures.
}
\setlength{\tabcolsep}{12pt}
\begin{tabular}{llc}
\toprule
Model parameter        &                         & Posterior median [\text{UI}_{68\%}] \\
\cmidrule{1-3}
sampling efficiency    & age 1                   & 0.003 [0.001, 0.005]                \\
&                        age 2                   & 0.13 [0.08, 0.19]                   \\
&                        age 3                   & 0.47 [0.30, 0.66]                   \\
&                        age 4+                  & 0.66 [0.46, 0.83]                   \\
zero-inflation rate    &                         & 0.34 [0.32, 0.35]                   \\
random walk SD         & survival probability    & 1.94 [1.71, 2.23]                   \\
&                        recruitment capita^{-1} & 1.45 [1.25, 1.68]                   \\
\bottomrule
\end{tabular}
\end{table}
\clearpage
% ---------------------------------------------------------------------------------------

% ---------------------------------------------------------------------------------------
\clearpage
\begin{table}
\caption{\label{tab:gls}
Coefficients and standard errors (SE) from the GLS models 
quantifying linear trends in the demographic rates through time.
The models were fit on either a log (recruitment; population growth rate)
or logit (survival probability) scale,
and all response and predictor variables were z-transformed prior to model fitting.
Therefore, the coefficients can be compared as effect sizes 
across all of the response variables.
The autoregressive (AR) coefficients were included to account for temporal autocorrelation
when estimating the year trends;
0 indicates no autocorrelation, 
while |1| indicates strong (positive or negative) autocorrelation 
and statistical non-stationarity.
}
\setlength{\tabcolsep}{12pt}
\begin{tabular}{llccc}
\toprule
Response variable       &        & Intercept (\pm~\text{SE}) & Year slope (\pm~\text{SE}) & AR
                                                                                 coefficient \\
\cmidrule{1-5}
survival probability    & age 1  & ~0.00  \pm~0.27           & ~0.15 \pm~0.26        & ~0.41 \\
&                         age 2  & -0.01  \pm~0.15           & ~0.27 \pm~0.16        & -0.10 \\
&                         age 3  & ~0.01  \pm~0.53           & ~0.19 \pm~0.43        & ~0.80 \\
&                         age 4+ &  0.04  \pm~0.25           & ~0.57 \pm~0.24        & ~0.49 \\
recruitment capita^{-1} &        & ~0.05  \pm~0.26           & -0.39 \pm~0.25        & ~0.45 \\
population growth rate  &        & -0.02  \pm~0.13           & ~0.18 \pm~0.13        & -0.27 \\
\bottomrule
\end{tabular}
\end{table}
\clearpage
% ---------------------------------------------------------------------------------------


% ---------------------------------------------------------------------------------------
\clearpage
\begin{figure}
\centering
\includegraphics{../analysis/figures/p_catch.pdf}
\caption{\label{fig:p_catch}
Station-level catch (points) and 90\% posterior prediction intervals (shading) 
from (a) the full model including temporal variation in survival and recruitment
and (b) the reduced model with survival and recruitment fixed through time.
The prediction intervals include stochasticity arising from the 
zero-inflated Poisson sampling process
and therefore represent the predicted distribution of 
station-level catch according to the model. 
The solid lines are medians of the posterior prediction distributions 
and quantify the expected catch; 
note that this is not the same quantity as the estimated population density, 
shown in Figure \ref{fig:p_dens} and Supporting Information: Figure S3
for the full and reduced models, respectively.
The y-axis is on a log+1 scale to accommodate zeros.
}
\end{figure}
\clearpage
% ---------------------------------------------------------------------------------------

% ---------------------------------------------------------------------------------------
\clearpage
\begin{figure}
\centering
\includegraphics{../analysis/figures/p_dens.pdf}
\caption{\label{fig:p_dens}
Average population density across stations (solid line) 
inferred from the demographic model.
Shading depicts 68\% uncertainty intervals, 
matching the nominal coverage of standard errors.
The y-axis is on a log scale.
}
\end{figure}
\clearpage
% ---------------------------------------------------------------------------------------

% ---------------------------------------------------------------------------------------
\clearpage
\begin{figure}
\centering
\includegraphics{../analysis/figures/p_lam.pdf}
\caption{\label{fig:p_lam}
Asymptotic population growth rate (solid blue line) inferred from the demographic model.
Shading depicts 68\% uncertainty intervals, 
matching the nominal coverage of standard errors.
The dashed horizontal line indicates a growth rate of 1,
which corresponds to no change in the population size from one time step to the next.
The solid black line represents the fitted values from the GLS model
fit on a log scale and then back-transformed 
to match the scale of the population growth rate.
Note, however, that the y-axis is on a log scale, 
thereby preserving the linearity of the data and model fit.
}
\end{figure}
\clearpage
% ---------------------------------------------------------------------------------------

% ---------------------------------------------------------------------------------------
\clearpage
\begin{figure}
\centering
\includegraphics{../analysis/figures/p_surv.pdf}
\caption{\label{fig:p_surv}
Logit-survival probability (solid colored lines) inferred from the demographic model.
Shading depicts 68\% uncertainty intervals, 
matching the nominal coverage of standard errors.
The solid black line represents the fitted values from the GLS models
fit separately for each age class.
Note that logit-survival probability ranging from -5 to 5 
corresponds to actual survival probability ranging from 0.007 to 0.993.
The dashed horizontal line indicates a zero, corresponding to 
an actual survival probability of 0.5.
}
\end{figure}
\clearpage
% ---------------------------------------------------------------------------------------

% ---------------------------------------------------------------------------------------
\clearpage
\begin{figure}
\centering
\includegraphics{../analysis/figures/p_rec.pdf}
\caption{\label{fig:p_rec}
Log-recruitment $\text{capita}^{-1}$ (solid blue line) inferred from the demographic model.
Shading depicts 68\% uncertainty intervals, 
matching the nominal coverage of standard errors.
The solid black line represents the fitted values from the GLS models.
Note that log-recruitment $\text{capita}^{-1}$ ranging from -1 to 8 
corresponds to actual log-recruitment $\text{capita}^{-1}$ ranging from 0.4 to 3000.
}
\end{figure}
\clearpage
% ---------------------------------------------------------------------------------------

\end{document}

