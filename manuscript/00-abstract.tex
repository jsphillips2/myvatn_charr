
\section*{Abstract} \label{abstract}

Quantifying temporal variation in demographic rates is a central goal of population ecology,
in both basic and applied settings.
In this study, we analysed a multidecadal age-structured time series 
of arctic charr (\emph{Salvelinus alpinus}) catch  in Lake M\'{y}vatn, Iceland, 
to infer the time-varying demographic response of the population 
to reduced harvest in the wake of the fishery's collapse.
Our analysis shows that while survival probability of adults increased following the alleviation
of harvesting pressure, per recruitment consistent declined over the entire study period.
The countervailing demographic trends resulted 
in no directional change in the total population size or population growth rate.
Rather, the population dynamics were dominated by large interannual variability
and a shift towards a relatively older age distribution.
These results are indicative of a slow recovery of the population following its collapse,
despite the rising number of adults due to relaxed harvest.

