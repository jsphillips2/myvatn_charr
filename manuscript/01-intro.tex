\section*{Introduction}

Quantifying temporal variation in demographic rates is a central goal of population ecology,
as this underpins efforts to characterize both exogenous and endogenous drivers of population dynamics 
\citep{twombly1994comparative, zeng1998, koons2016life}.
However, this endeavor is challenging, 
as even the characterization of directional trends in demographic rates requires
data spanning many years and 
often relies on intensive mark-recpature (and related) approaches for statistical inference
\citep[e.g.,][]{forcada2008life, hunter2010climate}.
Both of these challenges are amplified in populations subject to large interannual variation
in demographic rates, 
which are often the targets of both basic and applied interest
\citep[e.g.,][]{white2007irruptive}.
Consequently, 
there is a need for addtional studies that 
explict quantify temporal variation in demographic rates 
and the resulting population dynamics, 
particularly for populations that have not been the subjects 
of intensive mark-recpature-style campaigns.

In this study, we analysed a multidecadal age-structured time series 
of arctic charr (\emph{Salvelinus alpinus}) catch  in Lake M\'{y}vatn, Iceland, 
to infer the time-varying demographic response of the population 
to reduced harvest in the wake of the fishery's collapse.
Arctic charr are salmonids with a circumpolar distribution
and are basis of numerous commerical fisheries \citep{klemetsen2003atlantic}.
Moreover,
arctic charr have important effects in many artic 
and boreal freshwater foodwebs owing to their roles as top consumers
\citep{jeppesen2001fish, klemetsen2003atlantic}.
In M\'{y}vatn, the charr sustained a large commerical fishery throughout much of the 
twentieth century that was subject to exploitation rates upwards of 80\%,
resulting in the fishery's collapse by the late 1980s \citep{gudbergsson2004}.
A monitoring program was instituted in 1986 to monitor the population's
response to declining harvesting pressure over the next few decades,
including catch restrictions imposed in the early 2000s.

Using these monitoring data, 
we parameterized an age-structured demographic model with time-varying recruitment and survival 
to charactreize the dynamics of the population.
We modeled temporal variation in the demgoraphic rates as random walks, 
an approach that had previously been applied 
to infer population growth rates from non-structured abundance 
\citep{zeng1998}
and for age-specific mortality rates inferred from fisheries stock assessments 
\citep{nielsen2014estimation}. 
This method takes advantage of the entire time series for estimating the parameters 
while allowing them to vary smoothly through time. 
Furthermore, the model is able to characterize a range of dynamics including those
arising from negative density-dependence and environmental perturbations.
The latter may be particularly important in the case of the M\'{y}vatn's charr,
as the lake is subject to large fluctuations in primary and secondary production
that may cascade up to the charr population 
\citep{einarsson2004myvatn, einarsson2004clad, gardarsson2004population}.
Accordingly, the purpose of our analysis is to 
(a) characterize interannual variability in survival recruitment 
in M\'{y}vatn's charr population 
and (b) determine whether directional trends in these demographic rates have
resulted in the recovery of the population in the wake of its collapse. 