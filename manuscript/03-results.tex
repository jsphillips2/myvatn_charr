% ---------------------------------------------------------------------------------------
\section*{Results}
% ---------------------------------------------------------------------------------------

The full model provided a good visual fit to the station-level 
catch data (Figure \ref{fig:p_catch}),
with most of the temporal variation in lake-wide catch being captured by the model.
The 90\% prediction intervals provided reasonable coverage relative to the observed data,
although the observations were zero-inflated in some years.
While in principle it would be possible to account for such zero inflation in the model,
it is unlikely that this would substantively alter the model inference.
In contrast to the full model,
the reduced model provided a poor fit to the data (Figure \ref{fig:p_catch_reduced}),
displaying damped oscillations that did not reflect the observed dynamics
(Figure \ref{fig:p_dens_reduced}).
This was corroborated the median posterior log-likelihood calculated
from equation \ref{eq:likelihood},
which was much higher for the full model (-4826) than for the reduced (-7314).
According to the full model, 
the catch rates increased with age;
the catch rate for age 1 individuals was around two orders of magnitude lower
than for the other age classes (Table \ref{tab:param}). 
This is consistent with the expectation that age 1 individuals 
are generally too small to be captured by the gill nets used in the surveys.

M\'{y}vatn's arctic charr population fluctuated substantially over the 3-decade time series
(Figure \ref{fig:p_dens}), 
with the asymptotic population growth rate varying across an order of magnitude 
(Figure \ref{fig:p_lam}).
This variation was underpinned by substantial variation in survival 
(Figure \ref{fig:p_surv}) 
and recruitment (Figure \ref{fig:p_rec}),
both of which had random walk SDs 
with posterior densities concentrated away from zero (Table \ref{tab:param}).
The SD for survival as larger than for recruitment,
especially when the SDs were judged against the scales of their corresponding 
demographic processes. 
This indicates that survival was generally more variable than recruitment,
a fact that was visually apparent from their respective time series
(Figures \ref{fig:p_surv} and \ref{fig:p_rec}),
especially for age 1 and age 3 survival.

The survival probabilities for all age classes increased through time,
although only the trend for age 4+ was statistically unambiguous 
(Figure \ref{fig:p_surv}; Table \ref{tab:gls}).
Elevated age 4+ survival was associated with a steady increase in the catch of
age 4+ individuals from 2005 onward (Figure \ref{fig:p_catch}). 
In contrast, per capita recruitment declined through time, 
consistent with declining age 1 catch,
and this trend was stronger than for survival of any age class
(Figure \ref{fig:p_rec}; Table \ref{tab:gls}).
Therefore, the positive effect of increased survival 
was potentially negated by the decline in recruitment.
Indeed, the geometric mean of the population growth rate across years was very close to one
(0.99 [0.970, 1.02]), indicating no long-term population change.
Furthermore, there was no clear trend in the population growth rate itself,
despite its large interannual fluctuations
(Figure \ref{fig:p_lam}; Table \ref{tab:gls}).
The AR coefficients were moderately positive for most of the demographic parameters,
except for age 2 survival and the population growth rate 
for which the AR coefficients were slightly negative.
These negative coefficients potentially indicate overcompensatory dynamics,
although this inference is quite tentative given their low magnitudes 
and absence of uncertainty estimates.
Overall, the GLS results show that while the arctic charr population was very dynamic
over the study period, it did not undergo meaningful directional change.