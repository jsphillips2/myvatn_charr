% ---------------------------------------------------------------------------------------
\section*{Results}
% ---------------------------------------------------------------------------------------

The full model provided a good visual fit to the station-level 
catch data (Figure \ref{fig:p_catch}),
with most of the variation in average catch being captured by the model.
The 90\% prediction intervals provided reasonable coverage relative to the observed data,
although the latter were slightly zero-inflated in some years.
While in principle it would be possible to account for such zero inflation,
it is unlikely that this would substantively alter the model inference.
In contrast to the full model,
the reduced model provided a poor fit to the data (Figure \ref{fig:p_catch_reduced}),
displaying damped oscillations that did not reflect the observed dynamics
(Figure \ref{fig:p_dens_reduced}).
This was corroborated the median posterior log-likelihood calculated
from equation \ref{eq:likelihood},
which was much higher for the full model (-4826) than for the reduced (-7314).
According to the full model, 
the detection probabilities increased with age,
with age 1 individuals having a catch probably an order of magnitude lower
than the other age classes (Table \ref{tab:param}).


M\'{y}vatn's arctic charr population fluctuated substantially over the 3-decade time series
(Figure \ref{fig:p_dens}), 
resulting from interannual variation in survival (Figure \ref{fig:p_surv}) 
and recruitment (Figure \ref{fig:p_rec}) 
as indicated by the random walk standard deviations (Table \ref{tab:param}).
The survival probabilities for all age classes increased through time,
although only the trend for adults was unambiguous (Table \ref{tab:gls}). 
In contrast, per capita recruitment declined through time, 
potentially counteracting the effects of increased adult survival.
Indeed, the geometric mean of the asymptotic growth rate across years was very close to one
(0.99; [0.970, 1.02]), indicating no long-term population change.
Furthermore, there was no clear trend in the population growth rate itself,
despite its large interannual fluctuations.
Together, these results show that while the arctic charr population was very dynamic
over the study period, it did not undergo meaningful directional change.

