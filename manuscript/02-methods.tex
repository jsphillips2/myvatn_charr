% ---------------------------------------------------------------------------------------
\section*{Methods} 
% ---------------------------------------------------------------------------------------

% ---------------------------------------------------------------------------------------
\subsection*{Study system} 
% ---------------------------------------------------------------------------------------

Mývatn is located in northeastern Iceland (65°40’N 17°00’W) 
and has a tundra-subarctic climate (Björnsson and Jónsson 2004). 
The lake is 37 km2, divided into north (8 km2) and south (29 km2) basins 
connected by a narrow channel (Einarsson et al. 2004). 
Mývatn is shallow (mean depth = 2.3 m) and is fed by cold and 
warms springs rich in phosphorous 
(south basin N and P inputs of 1.4 and 1.5 g m-2 y-1, respectively) (Ólafsson 1979). 
Consequently, the lake has high primary production 
(Ólafsson 1979) that supports large but variable populations of benthic invertebrates 
such as midges and cladocerans 
(Lindegaard and Jónasson 1979, Einarsson and Örnólfsdóttir 2004, Garðarsson et al. 2004). 
The benthic invertebrates are an important food source for Mývatn’s vertebrate populations, 
including Arctic charr, three-spined sticklebacks (Gasterosteus aculeatus), 
brown trout (Salmo trutta), and waterfowl (Einarsson et al. 2004). 

Arctic charr (Salvelinus alpinus) are salmonids with a circumpolar distribution 
and are important consumers in many boreal and arctic lakes (Klemetsen et al. 2003). 
While some charr populations are anadromous (Klemetsen et al. 2003), 
Mývatn’s population to resides strictly within the lake, 
despite a major outflow that connects the lake to the Greenland Sea (Guðbergsson 2004). 
Spawning occurs in the fall among individuals aged 4 years and older. 
There are two charr morphs within Mývatn: “regular” (size at maturation of 35-50cm) 
and “dwarf” (20-25cm). However, the dwarf morph is restricted to a small 
and relatively isolated part of the south basin; therefore, 
the “regular” morph is the focus of this study. 
Gut content data reveal a diverse diet, including midges, snails, clams, zooplankton, 
benthic crustaceans, and sticklebacks. 
The large-bodied cladoceran Eurycercus is thought to be a particularly important prey item 
(Guðbergsson 2004). 
While some charr populations are cannibalistic (Klemetsen et al. 2003), 
especially in the absence of other large-bodied prey (e.g., fish), 
the charr in Mývatn appear to lack cannibalism. 

% ---------------------------------------------------------------------------------------
\subsection*{Data} 
% ---------------------------------------------------------------------------------------

Systematic surveys of Mývatn’s Arctic charr have been conducted 
with gill nets every year from 1986-2017 by a single researcher (G. Guðbergsson) 
using a consistent methodology. 
The surveys largely took place after the period of most dramatic population decline 
(Guðbergsson 2004). Most captured individuals were measured for length, mass, 
age (either based on otoliths or estimated from size), and a binary index of maturity. 
Surveys were conducted at 12 sites around the lake, 
most of which were located in the south basin. 
Surveys were conducted in fall (late August through September) 
of every year (just before spawning) and in June in a subset of years. 
For this analysis we used only the September data. 
Therefore, in our age classifications an individual of age ‘x’ is an individual 
that survived to that age (e.g., an “age 1” individual was born in the previous fall). 
The oldest individuals in the data set were 12 years old, although observations 
of individuals older than 6 years are sparse (Figure S2). 
In Mývatn, charr reach maturity between 4 and 5 years of age, 
and therefore we grouped the older individuals into a single reproductive age class, 
called “adults” or “age 4” for simplicity.

% ---------------------------------------------------------------------------------------
\subsection*{Demographic model} 
% ---------------------------------------------------------------------------------------

We characterized the demography of the charr population using an age-structured model
with time-varying demographic rates.
The model projected the dynamics from one time step to the next as
%
\begin{equation} \label{eq:XPX}
    \mathbf{x}_t = \mathbf{P}_{t-1}~\mathbf{x}_{t-1}
\end{equation}
%
where $\mathbf{P}_{t}$ is a demographic projection matrix
and $\mathbf{x}_t$ is an age-structured vector of scaled population densities (see below)
at time $t$.
%
The projection matrix was defined as
%
\begin{equation} \label{eq:matrix}
\mathbf{P}_{t} = 
\left[
\begin{array}{cccccccc}
    0             & 0             & 0             & \rho_{t}     \\
    \phi_{1,t}    & 0             & 0             & 0            \\
    0             & \phi_{2,t}    & 0             & 0            \\
    0             & 0             & \phi_{3,t}    & \phi_{4,t}
    \end{array}
\right]
\end{equation}
%
where $\rho_{t}$ is per capita recruitment 
and $\phi_{1,t}$ is the survival probability of age class $i$. 
The model assumes that only individuals of age four or greater reproduce
and allows adults return to the adult age class.
%
Temporal variation in recruitment was modeled as a random walk on a log scale to ensure
that the values remained positive:
%
\begin{equation} \label{eq:rho}
\begin{aligned}
\text{log}\left(\rho_t\right) &= \text{log}\left(\rho_{t - 1}\right) + \epsilon_{\rho,t} \\
\epsilon_{\rho,t} &\sim \mathematical{N}\left(0, \sigma_{\rho} \right)
\end{aligned}
\end{equation}
%
where $\epsilon_{\rho,t}$ is the random walk step at time $t$.
The steps of the random walk followed a Gaussian distribution with mean 0 
and standard deviation $\epsilon_{\rho,t}$.
Temporal variation in survival probability for age class $i$ was modeled analogously 
on a logit-scale:
%
\begin{equation} \label{eq:phi}
\begin{aligned}
\text{logit}\left(\phi_{i, t}\right) &= \text{logit}\left(\phi_{i, t - 1}\right) +
                                          \epsilon_{\phi,i,t} \\
\epsilon_{\phi,i,t} &\sim \mathematical{N}\left(0, \sigma_{\phi} \right)\text{.}
\end{aligned}
\end{equation}
%
Note that the ``random walks'' used to characterize the demographic parameters 
do not necessarily possess the statistical properties 
of true random walks (e.g., non-stationarity) 
as they are constrained by the data during model fitting.
Rather, they provided a convenient means 
of allowing the parameters to vary smoothly through time.

We parameterized the demographic model in a Bayesian framework, with likelihood
%
\begin{equation} \label{eq:likelihood}
\mathcal{L} = 
\displaystyle\prod_{s}
\displaystyle\prod_{i}
\displaystyle\prod_{t}
\text{Poisson}
    \left(
        y_{s,i,t}~|~\kappa_i~\times~s~\times~x_{i,t}
    \right)
\end{equation}
%
where $y_{s,i,t}$ is the annual catch for station $s$ and age class $i$,
$\kappa_i$ is the age-specific detection probability,
and $s=100$ is a scaling factor introduced to improve computationally efficiency
by preventing exessively large values of $x_{i,t}$ 
(such that $x_{i,t}$ itself is the population density scaled by $1/s$).
The likelihood implies that the station-level abundance follows a Poisson distribution
with a rate parameter equal to the lake-wide mean population density 
scaled by the catch probability for the corresponding age class.
We assumed that the lake-wide population was well mixed across years,
and while we did not attempt to account for systematic differences between stations
no such differences were apparent from visual inspection of the data 
(see discussion of overdispersion below).
Estimating a separate detection probability for each age class allowed the model to account
for systematic differences in catch rates as a function of age.
Such differences are quite apparent in the data, 
as far too few first year individuals are captured to account for the abundance subsequent
ages, assuming a closed population. 
Our model is superficially similar to an N-mixture model, 
however it does not include an explicit binomial detection process of discrete latent states.
We opted for the simpler approach because we encountered substantial convergence problems
when attempting to fit a ``proper'' N-mixture model to these data.
We used exponential priors with rate 1/50 for the initial scaled population densities for
each age class, 
Gaussian priors with mean 0 and standard deviation of either 5 or 1 for 
initial log-recruitment or logit-survival (repsectively), 
Gamma priors with shape and scale both of 1.5,
and Beta priors with shape parameters both of 2 for the detection probability.
Priors were selected to be weakly informative regarding overall scale, 
while not acting strongly against the likelihood in the posterior fit.

To assess temporal trends in the demographic rates, 
we used generalized least squares  (GLS) to fit lag-1 autoregressive models
with linear year effects.
Recruitment and survival probabilities were fit on their corresponding link scales 
(either log or logit),
and survival probabilities were fit for each age class separately.
We used a similar approach to calculate trends in the asymptotic population growth rate
$\lambda_t$, caclulated as the leading eigenvalue of projection matrix $\mathbf{P}_{t}$.
We performed this analysis on the asymptotic (as opposed to realized) growth rate 
as we were principally interersted in the effects 
of the underlying demographic rates per se; 
the realized growth rate would also have included transient fluctuations due to
non-equilibrium age structure.
For all GLS models, the response variables and year were z-scored so that coefficients
could be interpreted as effects sizes;
fitted values were then back-scaled for the figures.

The demographic model was fit using the software Stan through the \texttt{rstan} package,
with 4 chains, 3000 iterations, maximumum tree depth set to 13, and adapt delta set to 0.95.
Convergence and quality of MCMC sampling were assessed using the diagnostics provided by Stan, 
incuding Rhat, the number of divergences, and the effective sample size.
We used posterior medians as point esitmates 
and the bounds of 68\% posterior quantiles as uncertainty intervals 
(hereafter $\text{UI}_{68\%}$), 
matching the coverage of standard errors.
To gauge the degree to which the data were overdispersed relative 
the the model expectations, 
we simulated 90\% prediction intervals from equation \ref{eq:likelihood} 
and compared the coverage of these intervals to the data.
To assess the extent to which the data contained information regarding temporal variation 
in the demographic rates,
we compared the fit of the full model to a reduced model with the demographic
rates fixed through time.
We performed GLS using the \texttt{gls} function from the \texttt{nlme} package 
and calculated the asymptotic growth rate using the \texttt{demogR} package.
All analysis were conducted in R 4.0.3. 

