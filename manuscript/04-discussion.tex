
\section*{Discussion}

In this study, 
we used data from a multidecadal survey of arctic charr in Lake M\'{y}vatn
to quantify the population's demography and potential recovery 
following its collapse due to heavy exploitation 
\citep{gudbergsson2004}.
The survival probability of all age classes fluctuated substantially among years,
with only adults showing an unambiguous positive trend over the course of the study period.
In contrast, per capita recruitment clearly declined and experienced comparatively little
variation around this trend.
The countervailing changes in  per capita recruitment and adult survival resulted 
in no directional change in the total population size or population growth rate,
despite the rising number of adults.
Rather, the dynamics of  M\'{y}vatn's charr population 
were dominated by large interannual variability
and a shift towards a relatively older age distribution.
In and of itself, the increase in adult survival 
is a positive signal for the population's recovery,
and it has been taken as such by the local stakeholders 
who have called for relaxing harvest restrictions in recent years.
However, persistently low recruitment provides a cautionary note 
that should be taken into consideration in formulating a management strategy.

A crucial step for projecting the future dynamics of M\'{y}vatn’s charr population 
is identifying the underlying causes for the decline in per capita recruitment.
Food web interactions have been identified as an important source 
of fluctuations in other charr populations 
\citep{snorrason1992population, amundsen1994piscivory, jonsson2015freshwater}.
M\'{y}vatn is characterized by dramatic fluctuations in the abundance 
of primary food sources for juvenile charr, particularly benthic crustaceans 
and midges 
\citep{einarsson2004clad, gardarsson2004population, gudbergsson2004}.
Furthermore, there is substantial spatial heterogeneity in the abundance 
of these aquatic invertebrates \citep{bartrons2015spatial} 
which might disproportionately inhibit young juveniles 
that have more restricted mobility than the larger age classes. 
The large fluctuations in aquatic invertebrates are associated various consumer species 
in addition to charr, including sticklebacks and brown trout that could serve 
as competitors for young charr, and piscivorous waterfowl that could serve as predators 
\citep{einarsson2004myvatn}. 
In addition to biotic factors, 
temperature has received much attention as a driver of charr populations 
\citep{winfield2008arctic, elliott2010temperature, jonsson2015freshwater},
given their distribution restricted to arctic
and cold-temperature lakes \citep{klemetsen2003atlantic}
and the ubiquity anthropogenic climate change. 
While climate warming has not yet become an obvious ecological driver at M\'{y}vatn, 
it is nonetheless possible that temperature changes have adversely affected recruitment 
in M\'{y}vatn’s charr population as has been seen in other Icelandic lakes 
\citep{malmquist2009salmonid}. 

This study provides a demographic assessment of a population of arctic charr in a single lake,
but it reinforces themes that have broad and increasing interest in applied ecology.
Harvest-induced shifts in age structure have been documented in other fisheries,
typically resulting in ``truncation'' of the oldest age classes 
that are typically the targets of harvest efforts
\citep{hsieh2010fishing}.
Suppression of adult abundance is expected to have deleterious effects on populations,
with the corollary that relaxation of harvest should allow populations 
to recover following overexploitation.
However, this will only be true if juvenile recruitment is sufficient 
to sustaining the population's recovery.
Recruitment in fish populations 
has long been recognized as highly variable and difficult to predict
owing to the interplay of numerous biotic and abiotic factors 
\citep{dixon1999episodic, houde2008emerging, ludsin2014physical}.
This poses a particular challenge for management efforts, 
as complex suites of ecological factors are both difficult to understand 
and difficult to regulate, 
particularly in comparison to a direct anthropogenic driver such as harvest
\citep{beamish1999taking, link2002does}.
The extent to which this is true for M\'{y}vatn's arctic charr is currently unknown.
Nonetheless, 
the countervailing trends in survival and recruitment 
in the wake of alleviated of harvesting pressure
underscore the potential for heterogeneous demographic responses to management efforts
due to the complex ecological context in which such efforts take place.

