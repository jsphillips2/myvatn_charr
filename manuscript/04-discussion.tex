
\section*{Discussion}

In this study, 
we used data from a 3-decades-long survey of arctic charr in Lake M\'{y}vatn
to quantify the population's demography and potential recovery 
following its collapse due to heavy exploitation 
\citep{gudbergsson2004}.
The survival probability of all age classes fluctuated substantially among years,
with only adults showing a systematic positive trend over the course of the study period.
In constrast, per capita recruitment clearly declined and experienced comparatively little
variation superimposed on this decline.
The countervailing trends in adult survival and per capita recruitment resulted 
in no directional change in the total population size or population growth rate,
despite rising number of adults.
Rather, the dynamics of  M\'{y}vatn charr are dominated by large interannual variability
and a shift towards relatively older age distribution.
Harvest-induced shifts in age structure are well known in fisheries biology,
as is their capicity to limit recovery following overexploitation 
\citep{hsieh2010fishing}.

A crucial step for projecting the future dynamics of M\'{y}vatn’s charr population 
is identifying the underlying causes for changes in per capita recruitment, 
which is a major goal of fisheries biology more generally 
\citep{houde2008emerging, ludsin2014physical, hansen2013rapid}.
Food web interactions have been identified as an important source 
of fluctuations in other charr populations 
\citep{snorrason1992population, amundsen1994piscivory, jonsson2015freshwater}.
M\'{y}vatn is characterized by dramatic fluctuations in the abundance 
of primary food sources for juvenile charr, particularly benthic crustaceans 
and midges 
\citep{einarsson2004clad, gardarsson2004population, gudbergsson2004}.
Furthermore, there is substantial spatial heterogeneity in the abundance 
of these aquatic invertebrates \citep{bartrons2015spatial} 
which might disproportionately inhibit young juveniles 
that have more restricted mobility than the larger age classes. 
The large fluctuations in aquatic invertebrates are associated various consumer species 
in addition to charr, including sticklebacks and brown trout that could serve 
as competitors for young charr, and piscivorous waterfowl that could serve as predators 
\citep{einarsson2004myvatn}. 
In addition to biotic factors, 
temperature has received much attention as a driver of charr populations 
\citep{winfield2008arctic, elliott2010temperature, jonsson2015freshwater}
(Winfield et al. 2008, Elliott and Elliott 2010, Jonsson and Setzer 2015), 
given their distribution restricted to arctic
and cold-temperature lakes \citep{klemetsen2003atlantic}
and the ubiquity anthropogenic climate change. 
While climate warming has not yet become an obvious ecological driver at M\'{y}vatn, 
it is nonetheless possible that temperature changes have adversely affected recruitment 
in M\'{y}vatn’s charr population as has been seen in other Icelandic lakes 
\citep{malmquist2009salmonid}. 